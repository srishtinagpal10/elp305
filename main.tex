\documentclass[a4paper,12pt]{article}

% Packages for formatting and functionality
\usepackage[utf8]{inputenc} % Input encoding
\usepackage[T1]{fontenc}   % Font encoding
\usepackage{amsmath, amssymb} % Math symbols
\usepackage{graphicx}      % Including images
\usepackage{hyperref}      % Hyperlinks
\usepackage{geometry}      % Page layout
\usepackage{titlesec}      % Section formatting
\usepackage{booktabs}      % Tables
\usepackage{longtable}     % Long tables
\usepackage{xcolor}        % Text colors
\usepackage{float}         % Positioning of figures and tables
\geometry{margin=1in}      % Margins

% Hyperlink setup
\hypersetup{
    colorlinks=true,
    linkcolor=blue,
    filecolor=magenta,  
    urlcolor=cyan,
    pdftitle={Function Generator Report},
    pdfpagemode=FullScreen,
}

% Document Metadata
\title{\textbf{Function Generator Project Report}}
\author{\textit{Thursday Tribe}}
\date{\today}

\begin{document}

\maketitle
\tableofcontents
\newpage

% Section 1: Abstract
\section{Abstract}
This project introduces a function generator designed from scratch to produce precise periodic waveforms, including sinusoidal, square, and triangular signals, with adjustable frequency and amplitude. The device is tailored for use in academic laboratories, research environments, and small-scale industrial applications, offering a cost-effective and reliable alternative to commercial solutions.

Developed by a team of 65 undergraduate students, this project combines advanced electronic design principles, signal processing expertise, and practical functionality. The function generator integrates oscillators, waveform shaping circuits, and signal conditioning stages, along with a microcontroller-based interface for real-time control. Emphasis is placed on achieving high frequency stability, low total harmonic distortion, and user-friendly operation.

The device aims to bridge the gap between affordability and functionality, addressing the needs of users in resource-constrained settings. This paper details the design methodology, technological innovations, and potential impact of this versatile function generator on modern testing and prototyping environments.

\section{Motivation}
\textbf{Title:} \underline{Rationale for Designing a Cost-Effective and Versatile Function Generator}

The motivation for development of a function generator from scratch is driven by the need for a cost-effective, precise, and versatile signal generation solution tailored for academic, research, and small-scale industrial applications. Existing commercial function generators, while highly capable, often come with significant financial and operational constraints, limiting accessibility for resource-constrained environments.

Our team identified this gap through a comprehensive evaluation of current tools and their limitations. This project offers an opportunity to design a device that not only fulfills the essential requirements of waveform generation—producing sinusoidal, square, and triangular signals—but also ensures high frequency stability, low harmonic distortion, and user-friendly operation.

The project embodies our commitment to advancing practical engineering skills while addressing real-world challenges. By integrating principles of circuit design, signal processing, and digital control, the function generator aims to bridge the affordability gap without compromising on performance. This initiative is not merely an academic exercise but a step toward democratizing access to essential testing equipment, thereby empowering the broader engineering and scientific community.

% Section 2: Introduction
\section{Introduction}
Introduce the project, including:
\begin{itemize}
    \item Background and motivation.
    \item Objectives.
    \item Brief explanation of a function generator.
\end{itemize}

% Section 3: Design and Methodology
\section{Design and Methodology}
\subsection{Requirements}
Following is the list the technical and non-technical requirements of the function generator.
\subsubsection{Input Requirements}
\begin{itemize}
    \item \textbf{Waveform Control:}
    \begin{itemize}
        \item System must accept digital inputs for waveform selection (Sine, Square, Triangle, Ramp, and Pulse waves)
    \end{itemize}
    \item \textbf{Frequency Range:}
    \begin{itemize}
        \item User input for frequency selection from 1Hz to 10MHz through rotary encoder interface
    \end{itemize}
    \item \textbf{Amplitude Control:}
    \begin{itemize}
        \item Variable amplitude control from 0V to 5V peak-to-peak through user interface
    \end{itemize}
    \item \textbf{Phase Control:}
    \begin{itemize}
        \item 0-360 degrees phase adjustment capability when needed
    \end{itemize}
    \item \textbf{Power Input:}
    \begin{itemize}
        \item Since, the client hasn’t specified the power requirements. We assume the power supply must be standard power supply, which can be usually available at laboratory spaces
    \end{itemize}
    \item \textbf{User Interface:}
    \begin{itemize}
        \item Rotary encoder with push-button functionality for parameter selection and adjustment, and buttons for user selection
    \end{itemize}
    \item \textbf{Digital Control:}
    \begin{itemize}
        \item SPI interface between ESP32 and AD9833 for waveform generation
    \end{itemize}
\end{itemize}
\subsubsection{Output Requirements}
\begin{itemize}
    \item \textbf{Waveform Quality:}
    \begin{itemize}
        \item Clean output waveforms with minimal distortion (THD < 1\% for sine wave)
    \end{itemize}
    \item \textbf{Frequency Accuracy: }
    \begin{itemize}
        \item Generated frequency must be as close to the desired value as possible
    \end{itemize}
    \item \textbf{Amplitude Stability:}
    \begin{itemize}
        \item Output amplitude must remain stable as close as possible to the desired value
    \end{itemize}
    \item \textbf{Display Output:}
    \begin{itemize}
        \item Real-time display of frequency, waveform type, and amplitude settings on OLED/LCD
    \end{itemize}
    \item \textbf{Signal Output:}
    \begin{itemize}
        \item Industry-standard BNC connector with 50\(\Omega\) output impedance
    \end{itemize}
    \item \textbf{Noise Level:}
    \begin{itemize}
        \item Output signal-to-noise ratio to be kept as high as possible
    \end{itemize}
\end{itemize}
\subsubsection{Power Requirements}
\begin{itemize}
    \item \textbf{Input Power Source:}
    \begin{itemize}
        \item USB Type-C power delivery
        \item Standard mobile charger (5V)
        \item Power rating: 15W-20W capability recommended

    \end{itemize}
    \item \textbf{Protection:}
    \begin{itemize}
        \item Over-current protection for >2.5A
        \item Reverse polarity protection
        \item Short circuit protection
        \item Thermal protection

    \end{itemize}
    
    \item \textbf{Power Supply Requirements:}
    \begin{itemize}
        \item USB Type-C charger: 5V/3A or better
        \item Good quality power supply recommended
    \end{itemize}
\end{itemize}
    
\subsubsection{Environmental Requirements}
\begin{itemize}
    \item \textbf{Operating Conditions:}
    \begin{itemize}
        \item Temperature: +0°C to +40°C.
        \item Humidity: Up to 80\% relative humidity (non-condensing).
    \end{itemize}
    \item \textbf{Storage Conditions:}
    \begin{itemize}
        \item Temperature: -10°C to +70°C.
        \item Humidity: Up to 70\% relative humidity.
    \end{itemize}
    \item \textbf{Dust and Particle Resistance:}
    \begin{itemize}
        \item \textcolor{red}{TBD}
    \end{itemize}
\end{itemize}
\subsubsection{Site (Usage Site) Requirements}
\begin{itemize}
    \item \textbf{Location:}
    \begin{itemize}
        \item Designed for indoor use in laboratory or workshop environments.
    \end{itemize}
    \item \textbf{Space Requirements:}
    \begin{itemize}
        \item Compact design to fit on standard lab benches (dimensions: 20cm x 15cm x 5cm maximum).
    \end{itemize}
    \item \textbf{Power Supply:}
    \begin{itemize}
        \item Standard 5V power source availability.
    \end{itemize}
    \item \textbf{Ventilation:}
    \begin{itemize}
        \item Adequate airflow around the device to prevent overheating.
    \end{itemize}
    \item \textbf{Lighting:}
    \begin{itemize}
        \item Sufficient lighting for display readability.
    \end{itemize}
\end{itemize}

\subsubsection{Structural Requirements}
\begin{itemize}
    \item \textbf{Dimensions:}
    \begin{itemize}
        \item Maximum size of 200mm × 150mm × 100mm.
    \end{itemize}
    \item \textbf{Material:}
    \begin{itemize}
        \item \textbf{Enclosure:} 2mm aluminium sheets for durability and heat dissipation or some enclosure which provides shielding.
        \item \textbf{Front Panel:} 2mm acrylic sheet for labeling and controls.
    \end{itemize}
    \item \textbf{Components:}
    \begin{itemize}
        \item Rotary encoders (qty=2), rotary potentiometer (qty=1), BNC connectors (qty=1), push buttons (qty=5), square push button (qty=1).
    \end{itemize}
    \item \textbf{Assembly:}
    \begin{itemize}
        \item Modular design for easy assembly and disassembly.
        \item Rubber stoppers (qty=4) for stability and vibration damping.
        \item Bolts and nuts (qty=15 pairs) for secure fastening.
    \end{itemize}
    \item \textbf{Slidable Top Lid:}
    \begin{itemize}
        \item For easy access to internal components during maintenance.
    \end{itemize}
    \item \textbf{Access Points:}
    \begin{itemize}
        \item Easy access to calibration and test points.
    \end{itemize}
\end{itemize}

\subsubsection{Time Requirements}
\begin{itemize}
    \item \textbf{Design Time Requirement:}
    \begin{itemize}
        \item \textcolor{red}{??}
    \end{itemize}
    \item \textbf{Time to Market Requirement:}
    \begin{itemize}
        \item \textcolor{red}{??}
    \end{itemize}
    \item \textbf{Lifetime Requirement:}
    \begin{itemize}
        \item \textcolor{red}{??}
    \end{itemize}
    \item \textbf{End of Life Requirement:}
    \begin{itemize}
        \item \textcolor{red}{??}
    \end{itemize}
\end{itemize}

\subsubsection{Other Non-Functional Requirements
}
\begin{itemize}
    \item \textbf{Color:}
    \begin{itemize}
        \item Grey for a professional appearance.
    \end{itemize}
    \item \textbf{Weight:}
    \begin{itemize}
        \item Approximately 1 kg for portability.
    \end{itemize}
    \item \textbf{Safety:}
    \begin{itemize}
        \item Proper insulation to prevent electrical hazards
        \item No exposed wires or sharp edges.
        \item Grounded chassis for shock protection.
    \end{itemize}
    \item \textbf{Serviceability:}
    \begin{itemize}
        \item Modular design for easy repair and component replacement.
        \item Slidable top lid for quick access to internal parts.
    \end{itemize}
    \item \textbf{Reliability:}
    \begin{itemize}
        \item Stable performance under specific environmental conditions.
    \end{itemize}
    \item \textbf{Durability:}
    \begin{itemize}
        \item \textcolor{red}{???}
    \end{itemize}
\end{itemize}


\subsection{Block Diagram}
Include a labeled block diagram of the system:
\begin{figure}[H]
    \centering
    \includegraphics[width=0.8\textwidth]{block_diagram.png} % Replace with actual file
    \caption{Block Diagram of the Function Generator}
    \label{fig:block_diagram}
\end{figure}

\subsection{Circuit Design}
Explain the circuit design process, including schematics and reasoning for component selection.

\subsection{Implementation}
Discuss how the design was implemented, including tools and techniques used.

% Section 4: Results
\section{Results}
\subsection{Output Waveforms}
Include waveform images or data generated by the function generator:
\begin{figure}[H]
    \centering
    \includegraphics[width=0.8\textwidth]{output_waveform.png} % Replace with actual file
    \caption{Output Waveform Examples}
    \label{fig:output_waveform}
\end{figure}

\subsection{Performance Analysis}
Evaluate the performance of the function generator based on criteria like frequency range, amplitude, and signal stability.

% Section 5: Discussion
\section{Discussion}
Discuss the results and their implications. Highlight any challenges faced and how they were addressed. Suggest potential improvements or future work.



% Section 6: Who is the Single Point of Contact (SPOC)?
\section{Who is the Single Point of Contact (SPOC)?}
\begin{table}[H]
    \centering
    \begin{tabular}{|l|l|l|l|l|}
        \hline
        \textbf{Designation} & \textbf{Name} & \textbf{Entry No} & \textbf{Email} & \textbf{Phone} \\
        \hline
        Tribe Co-ordinator & Shahid  & & & \\
        \hline
        Deputy Tribe Co-ordinator &  & & & \\
        \hline
    \end{tabular}
    \caption{Single Point of Contact (SPOC)}
\end{table}
% Section 7: List All Tribe Members


\subsection{Documentation}
\begin{longtable}[c]{|l|l|l|l|l|l|l|}
\hline
\textbf{Designation} & \textbf{Name} & \textbf{Entry No} & \textbf{Email} & \textbf{Phone} & \textbf{IF (0 to 1)} & \textbf{Justification for Low IF} \\
\hline
Documentation & Prateek Mourya & 2022MT11937 & mt1221937@iitd.ac.in &  &  &  \\ \hline
Documentation & P Siddhartha & 2021EE10135 & ee1210135@iitd.ac.in &  &  &  \\ \hline
Documentation & Srishti Nagpal & 2022EE11165 & ee1221165@iitd.ac.in &  &  &  \\ \hline
Documentation & Ankita Malviya & 2022EE11183 & ee1221183@iitd.ac.in &  &  &  \\ \hline
Documentation & Kaustubh Vatsa & 2022EE11151 & ee1221151@iitd.ac.in &  &  &  \\ \hline
Documentation & Gaurav Gupta & 2022EE11691 & ee1221691@iitd.ac.in &  &  &  \\ \hline
Documentation & Mayank Kumar & 2022EE11727 & ee1221727@iitd.ac.in &  &  &  \\ \hline
Documentation & Nuthi Sai Kushwanth & 2021EE10144 & ee1210144@iitd.ac.in &  &  &  \\ \hline
Documentation & Abhinav Tiwari & 2022EE11665 & ee1221665@iitd.ac.in &  &  &  \\ \hline
Documentation & Deepanjan Mandal & 2022EE31784 & ee3221784@iitd.ac.in &  &  &  \\ \hline
Documentation & Pravakar Mohapatra & 2022EE31193 & ee3221193@iitd.ac.in &  &  &  \\ \hline
Documentation & Saumya Singh & 2022EE31785 & ee3221785@iitd.ac.in &  &  &  \\ \hline
Documentation & Repudi Niranjan Tagore & 2021EE10164 & ee1210164@iitd.ac.in &  &  &  \\ \hline
Documentation & Rachit Sharma & 2022EE31744 & ee3221744@iitd.ac.in &  &  &  \\ \hline
Documentation & Vaishnavi Nandkishor Bisen & 2022EE11719 & ee1221719@iitd.ac.in &  &  &  \\ \hline
Documentation & BHARAT AGARWAL & 2022EE11790 & ee1221790@iitd.ac.in &  &  &  \\ \hline
Documentation & Abheek Gera & 2022EE11160 & ee1221160@iitd.ac.in &  &  &  \\ \hline
\end{longtable}

\subsection{Electronics}
\begin{longtable}[c]{|l|l|l|l|l|l|l|}
\hline
\textbf{Designation} & \textbf{Name} & \textbf{Entry No} & \textbf{Email} & \textbf{Phone} & \textbf{IF (0 to 1)} & \textbf{Justification for Low IF} \\
\hline
Electronics & Kanika Jain & 2022EE11168 & ee1221168@iitd.ac.in &  &  &  \\ \hline
Electronics & Lavanya & 2022EE11679 & ee1221679@iitd.ac.in &  &  &  \\ \hline
Electronics & Bokam Praneeth & 2022EE11171 & ee1221171@iitd.ac.in &  &  &  \\ \hline
Electronics & Ashesh Mishra & 2022EE11155 & ee1221155@iitd.ac.in &  &  &  \\ \hline
Electronics & Arpit Prasad & 2022EE11837 & ee1221837@iitd.ac.in &  &  &  \\ \hline
Electronics & Advik Gupta & 2022EE31740 & ee3221740@iitd.ac.in &  &  &  \\ \hline
Electronics & Adheyan Gupta & 2022EE11659 & ee1221659@iitd.ac.in &  &  &  \\ \hline
Electronics & Anushka Chaturvedi & 2022EE31763 & ee3221763@iitd.ac.in &  &  &  \\ \hline
Electronics & Nakshat Pandey & 2022EE11436 & ee1221436@iitd.ac.in &  &  &  \\ \hline
Electronics & Krishna Kumar Gupta & 2022EE11698 & ee1221698@iitd.ac.in &  &  &  \\ \hline
Electronics & Tannu Shree & 2022EE11683 & ee1221683@iitd.ac.in &  &  &  \\ \hline
Electronics & Devansh Pandey & 2022EE31538 & ee3221538@iitd.ac.in &  &  &  \\ \hline
Electronics & Asmi Jayee & 2022EE31756 & ee3221756@iitd.ac.in &  &  &  \\ \hline
Electronics & Priya Jain & 2022EE31757 & ee3221757@iitd.ac.in &  &  &  \\ \hline
Electronics & Priyansh Dutt Sharma & 2022EE331739 & ee3221739@iitd.ac.in &  &  &  \\ \hline
\end{longtable}

\subsection{Mechanical}
\begin{longtable}[c]{|l|l|l|l|l|l|l|}
\hline
\textbf{Designation} & \textbf{Name} & \textbf{Entry No} & \textbf{Email} & \textbf{Phone} & \textbf{IF (0 to 1)} & \textbf{Justification for Low IF} \\
\hline
Mechanical & Chandan Solanki & 2022EE11726 & ee1221726@iitd.ac.in &  &  &  \\ \hline
Mechanical & Pulkit Sheoran & 2022EE11714 & ee1221714@iitd.ac.in &  &  &  \\ \hline
Mechanical & Sneha Dhaka & 2022EE11176 & ee1221176@iitd.ac.in &  &  &  \\ \hline
Mechanical & Akshat & 2022EE31768 & ee3221768@iitd.ac.in &  &  &  \\ \hline
Mechanical & Anushka Patel & 2022EE31776 & ee3221776@iitd.ac.in &  &  &  \\ \hline
Mechanical & Suryansh & 2022EE31431 & ee3221431@iitd.ac.in &  &  &  \\ \hline
Mechanical & Dhruv Jhunjhunwala & 2022EE32068 & ee3222068@iitd.ac.in &  &  &  \\ \hline
Mechanical & Arushi & 2022MT11950 & mt1221950@iitd.ac.in &  &  &  \\ \hline
Mechanical & Nikita & 2022MT11951 & mt1221951@iitd.ac.in &  &  &  \\ \hline
Mechanical & Janmesh Jarwal & 2022EE11186 & ee1221186@iitd.ac.in &  &  &  \\ \hline
Mechanical & Rajat Goswami & 2022EE31772 & ee3221772@iitd.ac.in &  &  &  \\ \hline
Mechanical & Kamal & 2022EE11703 & ee1221703@iitd.ac.in &  &  &  \\ \hline
Mechanical & Anubhav Karadia & 2022MT11292 & mt1221292@iitd.ac.in &  &  &  \\ \hline
Mechanical & Gajendra Dhanoliya & 2022EE11723 & ee1221723@iitd.ac.in &  &  &  \\ \hline
Mechanical & Lokendra Singh Gohil & 2022EE11164 & ee1221164@iitd.ac.in &  &  &  \\ \hline
Mechanical & Aditya Kumar Singh & 2022EE31783 & ee3221783@iitd.ac.in &  &  &  \\ \hline
Mechanical & Ayushya Joshi & 2022EE11674 & ee1221674@iitd.ac.in &  &  &  \\ \hline
Mechanical & Arnav Singhal & 2022EE11270 & ee1221270@iitd.ac.in &  &  &  \\ \hline
Mechanical & Priyanshu Mangawa & 2022EE11701 & ee1221701@iitd.ac.in &  &  &  \\ \hline
\end{longtable}

\subsection{Software}
\begin{longtable}[c]{|l|l|l|l|l|l|l|}
\hline
\textbf{Designation} & \textbf{Name} & \textbf{Entry No} & \textbf{Email} & \textbf{Phone} & \textbf{IF (0 to 1)} & \textbf{Justification for Low IF} \\
\hline
Software & Ankit Choudhary & 2022EE11707 & ee1221707@iitd.ac.in &  &  &  \\ \hline
Software & Dhruv Malhotra & 2022EE11670 & ee1221670@iitd.ac.in &  &  &  \\ \hline
Software & Rohit Rajput & 2022EE11699 & ee1221699@iitd.ac.in &  &  &  \\ \hline
Software & Simran Meena & 2022EE11188 & ee1221188@iitd.ac.in &  &  &  \\ \hline
Software & Pratham Malhotra & 2022EE11152 & ee1221152@iitd.ac.in &  &  &  \\ \hline
Software & Kabir Eshu Nagpal & 2022EE31743 & ee3221743@iitd.ac.in &  &  &  \\ \hline
Software & Shikhar Gupta & 2022MT11925 & mt1221925@iitd.ac.in &  &  &  \\ \hline
Software & Rohit Yuvaraj Jadekar & 2022MT61984 & mt6221984@iitd.ac.in &  &  &  \\ \hline
Software & Akshat Bhasin & 2022EE31996 & ee3221996@iitd.ac.in &  &  &  \\ \hline
\end{longtable}


% Section 8: Discussion
\section{Discussion}
Discuss the results and their implications. Highlight any challenges faced and how they were addressed. Suggest potential improvements or future work.

% Section 9: Conclusion
\section{Conclusion}
Summarize the key achievements of the project and its relevance.

% Section 10: References
\section{References}
Use Zotero or another reference manager to format citations in APA/IEEE/MLA style:
\begin{thebibliography}{99}
    \bibitem{example} Author Name, \textit{Title of Reference}, Publisher, Year.
\end{thebibliography}

% Section 11: Appendices
\appendix
\section{Appendices}
\subsection{Appendix A: Circuit Diagrams}
Additional circuit diagrams.

\subsection{Appendix B: Code}
Include any code used for generating waveforms or analyzing data.

\end{document}
