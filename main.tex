\documentclass[a4paper,12pt]{article}

% Packages for formatting and functionality
\usepackage[utf8]{inputenc} % Input encoding
\usepackage[T1]{fontenc}   % Font encoding
\usepackage{amsmath, amssymb} % Math symbols
\usepackage{graphicx}      % Including images
\usepackage{hyperref}      % Hyperlinks
\usepackage{geometry}      % Page layout
\usepackage{titlesec}      % Section formatting
\usepackage{booktabs}      % Tables
\usepackage{tabu}          % 
\usepackage{longtable}     % Long tables
\usepackage{xcolor}        % Text colors
\usepackage{float}         % Positioning of figures and tables
\geometry{margin=1in}      % Margins
\usepackage{changepage}


% Hyperlink setup
\hypersetup{
    colorlinks=true,
    linkcolor=blue,
    filecolor=magenta,  
    urlcolor=cyan,
    pdftitle={Function Generator Report},
    pdfpagemode=FullScreen,
}

% Document Metadata
\title{\textbf{Function Generator Project Report}}
\author{\textit{Thursday Tribe}}
\date{\today}

\begin{document}

\maketitle
\tableofcontents
\newpage

% Section 1: Abstract
\section{Abstract}
This project introduces a function generator designed from scratch to produce precise periodic waveforms, including sinusoidal, square, and triangular signals, with adjustable frequency and amplitude. The device is tailored for use in academic laboratories, research environments, and small-scale industrial applications, offering a cost-effective and reliable alternative to commercial solutions.

Developed by a team of 65 undergraduate students, this project combines advanced electronic design principles, signal processing expertise, and practical functionality. The function generator integrates oscillators, waveform shaping circuits, and signal conditioning stages, along with a microcontroller-based interface for real-time control. Emphasis is placed on achieving high frequency stability, low total harmonic distortion, and user-friendly operation.

The device aims to bridge the gap between affordability and functionality, addressing the needs of users in resource-constrained settings. This paper details the design methodology, technological innovations, and potential impact of this versatile function generator on modern testing and prototyping environments.

\section{Motivation}
\textbf{Title:} \underline{Rationale for Designing a Cost-Effective and Versatile Function Generator}

The motivation for development of a function generator from scratch is driven by the need for a cost-effective, precise, and versatile signal generation solution tailored for academic, research, and small-scale industrial applications. Existing commercial function generators, while highly capable, often come with significant financial and operational constraints, limiting accessibility for resource-constrained environments.

Our team identified this gap through a comprehensive evaluation of current tools and their limitations. This project offers an opportunity to design a device that not only fulfills the essential requirements of waveform generation—producing sinusoidal, square, and triangular signals—but also ensures high frequency stability, low harmonic distortion, and user-friendly operation.

The project embodies our commitment to advancing practical engineering skills while addressing real-world challenges. By integrating principles of circuit design, signal processing, and digital control, the function generator aims to bridge the affordability gap without compromising on performance. This initiative is not merely an academic exercise but a step toward democratizing access to essential testing equipment, thereby empowering the broader engineering and scientific community.
\end{document}